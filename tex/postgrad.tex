\documentclass[../main.tex]{subfiles}
 
\begin{document}

 %-----------PROJECTS-----------------
 \section{Java Projects}
 %DISSERTATION
 \begin{description}[style=multiline,leftmargin=4cm]
 	\item[Value at Risk \textnormal{Dissertation} \textnormal{\tiny
 		      \href{https://adrian.ng/java/var/}{adrian.ng/java/var/}}]
 	      Estimating \textit{VaR}, a measure of risk, for an investment portfolio containing stocks, options, deltas.
 	      \begin{minipage}[b]{0.2\textwidth}
 		      \vspace{0.25cm}
 		      %   \begin{description}[style=multiline,leftmargin=4cm]
 		      \textbf{VaR Measures}
 		      \begin{itemize}[leftmargin=*]
 			      \tiny
 			      \item Model Building
 			      \item Historical Simulation
 			      \item Monte Carlo Simulation.
 		      \end{itemize}
 		      \textbf{Volatility Estimates}
 		      \begin{itemize}[leftmargin=*]
 			      \tiny
 			      \item \textit{Equal Weighted}
 			      \item \textit{Equal Weighted Moving Average (EWMA)}
 			      \item \textit{GARCH(1,1)}
 		      \end{itemize}
 		      %   \end{description}
 		      %   \vspace{0.25cm}
 	      \end{minipage}
 	      \hspace{0.25cm}
 	      \begin{minipage}[b]{0.55\textwidth}
 		      \begin{itemize}[leftmargin=*]
 			      \item implemented \textit{Levenberg-Marquardt} algorithm for optimisation of \textit{GARCH(1,1)} parameters
 			      \item made use of object-oriented techniques and patterns
 			      \item gained efficiencies using Java's \texttt{concurrency} APIs to parallelize the 100,000+ random walks generated by \textit{Monte Carlo} when simulating stock price movements
 			      \item utilised \textit{Google Finance}/\textit{Yahoo Finance} APIs to source time-series financial data
 		      \end{itemize}
 	      \end{minipage}
 	      \dotfill
 	      %BIG DATA
 	\item[Data Mining \textnormal{Large-Scale Data Storage \& Processing} \textnormal{\tiny\href{https://github.com/Adrian-Ng/HadoopEnron}{adrian.ng/java/enron}} \textnormal{\tiny
 		      \href{https://adrian.ng/scala/enron1}{adrian.ng/scala/enron1}}]
 	      %   Wrote \textit{MapReduce} applications for large scale data mining and processing.
 	      \begin{description}[style=multiline,leftmargin=2.5cm]
 		      \item[MapReduce]
 		            Wrote \textit{MapReduce} applications involving:
 		            \begin{itemize}
 			            \item aggregation of \textit{Twitter} data, utilising \texttt{twitter4j} API
 			            \item scraping a large collection of emails in the \textit{Enron Corpus}
 			            \item extraction of communications graph consisting of nodes/edges
 		            \end{itemize}
 		      \item[Hadoop]
 		            \begin{itemize}
 			            \item Applications ran on both single-node (self-hosted)/ distributed-node clusters
 			            \item Interfaced with \textit{HDFS} via terminal command-line
 		            \end{itemize}
 		      \item[Spark] Utilised an \textit{Apache Spark REPL} to achieve:
 		            \begin{itemize}
 			            \item translation of \textit{MapReduce} applications to \texttt{Scala}
 			            \item reduced code verbosity
 			            \item ETL via \textit{HDFS} using \texttt{sparkcontext} API
 		            \end{itemize}
 	      \end{description}
 	      \dotfill
 	      %   \vspace{0.25cm}
 	\item[Option Pricing \textnormal{Methods of Computational Finance} \textnormal{\tiny\href{https://adrian.ng/java/options/}{adrian.ng/java/options/}}]
 	      Implemented numerous approaches to pricing options and calculating payoff:
 	      \begin{description}[style=multiline,leftmargin=2.85cm]
 		      \item[Options]
 		            \begin{itemize*}
 			            \item Monte Carlo Simulation
 			            \item Black Scholes
 			            \item Binomial Trees
 		            \end{itemize*}
 		      \item[Payoff]
 		            \begin{itemize*}
 			            \item American
 			            \item Asian
 			            \item European
 		            \end{itemize*}
 	      \end{description}
 	      These approaches made probabilistic assumptions, so \texttt{Apache Commons Math} API was used.
 	      \dotfill
 	\item[Summarizing financial data \newline \textnormal{\tiny
 		      \href{https://adrian.ng/java/yahoofinance/}{adrian.ng/java/yahoofinance/}}]
 	      An exercise in using Java 8's \texttt{Stream} API.
 	      I was able to implement approaches to computing mean and variance estimates from an immutable collection of time-series financial data.
 	      %   \begin{itemize}
 	      %       \item implmented approaches to computing statistics (mean, variance) on time-series financial data
 	      %   \end{itemize}
 	      %       \vspace{0.5cm}
 	      % \item[Google PageRank]
 	      %       Implementation of Google's \textit{PageRank} algorithm. I simulate the behaviour of someone browsing a series of webpages by computing a transition matrix from an input graph and mixing a Markov Chain.
 	      \dotfill
 	      % \item[Webpage Scraping]
 	      %       Wrote a program for scraping data from webpages.
 	      %       \begin{itemize}
 	      % 	      \item Utilised \texttt{HtmlUnit} and \texttt{Selenium} APIs
 	      % 	      \item Traversed DOM and parsed child elements via \texttt{xPath}
 	      %       \end{itemize}
 \end{description}

\end{document}