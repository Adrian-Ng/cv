%-------------------------
% Resume in Latex
% Author : Sourabh Bajaj
% License : MIT
%------------------------

\documentclass[letterpaper,11pt]{article}

\usepackage{latexsym}
\usepackage[empty]{fullpage}
\usepackage{titlesec}
\usepackage{marvosym}
\usepackage[usenames,dvipsnames]{color}
\usepackage{verbatim}
\usepackage{enumitem}
\usepackage[pdftex]{hyperref}
\usepackage{fancyhdr}


\pagestyle{fancy}
\fancyhf{} % clear all header and footer fields
\fancyfoot{}
\renewcommand{\headrulewidth}{0pt}
\renewcommand{\footrulewidth}{0pt}

% Adjust margins
\addtolength{\oddsidemargin}{-0.375in}
\addtolength{\evensidemargin}{-0.375in}
\addtolength{\textwidth}{1in}
\addtolength{\topmargin}{-.5in}
\addtolength{\textheight}{1.0in}

\urlstyle{same}

\raggedbottom
\raggedright
\setlength{\tabcolsep}{0in}

% Sections formatting
\titleformat{\section}{
	\vspace{-4pt}\scshape\raggedright\large
}{}{0em}{}[\color{black}\titlerule \vspace{-5pt}]

%-------------------------
% Custom commands
\newcommand{\resumeItem}[3]{
	\item\small{
		\textbf{#1}\hfill\tiny{#2\\}\small{ #3 \vspace{-2pt}}
	}
}

\newcommand{\resumeSubheading}[4]{
	\vspace{-1pt}\item
	\begin{tabular*}{0.97\textwidth}{l@{\extracolsep{\fill}}r}
		\textbf{#1} & #2 \\
		\textit{\small#3} & \textit{\small #4} \\
	\end{tabular*}\vspace{-5pt}
}

\newcommand{\resumeSubItem}[3]{\resumeItem{#1}{#2}{#3}\vspace{-2pt}}

\renewcommand{\labelitemii}{$\circ$}

\newcommand{\resumeSubHeadingListStart}{\begin{itemize}[leftmargin=*]}
	\newcommand{\resumeSubHeadingListEnd}{\end{itemize}}
\newcommand{\resumeItemListStart}{\begin{itemize}}
	\newcommand{\resumeItemListEnd}{\end{itemize}\vspace{-5pt}}

%-------------------------------------------
%%%%%%  CV STARTS HERE  %%%%%%%%%%%%%%%%%%%%%%%%%%%%


\begin{document}
	
	%----------HEADING-----------------
	\begin{tabular*}{\textwidth}{l@{\extracolsep{\fill}}r}
		\textbf{{\Large Adrian Ng MSc.}} & \textbf{Email:} \href{mailto:contact@adrian.ng}{contact@adrian.ng} \\
		Seeking Junior-Level Data Engineering Opportunities & \textbf{Website}: \href{https://adrian.ng}{adrian.ng} \\
	\end{tabular*}
	
	\section{Summary}
	
	I am a Computer Science graduate passionate about Data Engineering. I seek opportunities that further my growing experience in \textit{Java} -- which I have used in numerous academic projects ranging from the implementation of financial models to large-scale data processing with \textit{Apache Hadoop MapReduce}. 
	\newline
			
	Prior to postgraduate study, my expertise was in \textit{SQL development} focusing on the implementation of CRM segmentation processes for a number of clients including: \textit{Virgin Media}, \textit{TUI}, \textit{UPC}, \textit{MSD}, \textit{Volkswagen}, and \textit{KwikFit}. 
	\newline
		
	In my role as a Data Analyst at \textit{Manchester City FC}, I implemented end-to-end data pipelines (ingress, ETL, data cubes) for use in reporting dashboards. My strengths in this role were more technical than analytical -- which leads me now to pursue a career in programming.	
		 
	%-----------EDUCATION-----------------
	\section{Education}
	\resumeSubHeadingListStart
	\resumeSubheading
	{Master of Science in Data Science and Analytics}{with Distinction}
	{Department of Computer Science, Royal Holloway, University of London}{Sept. 2016 -- Dec. 2017}
	\resumeSubheading
	{Bachelor of Engineering in Mechanical Engineering}{Upper Second Class with Honours}
	{School of Engineering, King's College London, University of London}{Sept. 2007 -- July 2010}
	\resumeSubHeadingListEnd
	
%	\section{Postgraduate Modules}
%	\resumeSubHeadingListStart
%		\resumeSubheading
%		{Programming for Data Analysis}{}
%		{Java -- Eclipse etc}{}
%		\resumeSubheading
%		{Large Scale Data Storage and Processing}{}
%		{Java -- Apache Hadoop}{}
%		\resumeSubheading
%		{Methods of Financial Computation}{}
%		{Java, MATLAB}{}
%	\resumeSubHeadingListEnd
	
	%-----------PROJECTS-----------------
	\section{Java Projects}
	\resumeSubHeadingListStart
		\resumeSubItem
		{Implementation of Value at Risk (VaR) measures in Java}
		{
			(\href{https://adrian.ng/java/var/}{https://adrian.ng/java/var/}) \quad (\href{https://github.com/Adrian-Ng/VaR}{https://github.com/Adrian-Ng/VaR})
		}
		{  
			This dissertation project implemented various approaches to estimating \textit{VaR}, a measure of risk. These are: \textit{Model Building}, \textit{Historical Simulation}, and \textit{Monte Carlo Simulation}.
			In addition, the following approaches to estimating market variance/volatility were implemented: \textit{Equal Weighted}, \textit{Exponentially Weighted Moving Average}, and \textit{GARCH(1,1)}.
		}
		\resumeItemListStart
			\resumeItem{Object Oriented Design}{}
			{
				As we have a number of approaches to estimating both \textit{VaR}, \textit{variance} and \textit{volatility}, object oriented techniques and patterns were implemented.
			}
			\resumeItem{Concurrency}{}
			{
				The \textit{Monte Carlo} approach generates a large number of random walks, which can take a long time to fully execute in series. I used Java's concurrency API's to write a highly efficient solution.
			}
			\resumeItem{Data Ingress}{}
			{
				Real-world market data was sourced using the \textit{Yahoo Finance API}. These daily closing prices were transformed into daily price changes.
				\newline
				Other inputs included the parameters of our hypothetical investment portfolio (market assets, deltas).
			}
		\resumeItemListEnd
		
		\resumeSubItem{Option Pricing}
		{
			(\href{https://adrian.ng/java/options/}{https://adrian.ng/java/options/}) \quad (\href{https://github.com/Adrian-Ng/OptionPricer}{https://github.com/Adrian-Ng/OptionPricer})
		}
		{
			This project implements three approaches to estimating option prices in Java: \textit{Monte Carlo simulation}, \textit{Black-Scholes equations}, and \textit{Binomial Trees}.
			\newline
			Apache Commons Math API was used to deal with some probabilistic assumptions.
		}
		
		\resumeSubItem{Data Mining with Hadoop MapReduce}
		{
			(\href{https://github.com/Adrian-Ng/HadoopEnron}{https://github.com/Adrian-Ng/HadoopEnron})
		}
		{
			A number of \textit{MapReduce} applications were written in Java with a variety of purposes including extracting the communications network from the \textit{Enron Corpus}, a large dataset of emails, or aggregation of Twitter data.
			\newline
			Applications were exported and executed on \textit{Hadoop} clusters (both single node and distributed). Input/Output datasets were stored in HDFS and accessed via \texttt{hadoop fs} commands.
			\newline
			A subsequent exercise was undertaken to minimise the verbosity of these \textit{Hadoop MapReduce} applications by translating them to \textit{Scala} for use in a \textit{Spark REPL}.
		}
		
		\resumeSubItem{Java 8 Streams with financial data}
		{(\href{https://adrian.ng/java/yahoofinance/\#stream}{https://adrian.ng/java/yahoofinance/\#stream})}
		{
			A small exercise involving the use of \textit{Java 8 Streams}. Processing real-world financial data to return \textit{mean} and \textit{equal-weighted variance} of some market asset.
		}	
		 \resumeSubItem{Google PageRank}{}
		 {
		   This is the implementation of Google's \textit{PageRank} algorithm. I simulate the behaviour of someone browsing a series of webpages by computing a transition matrix from an input graph and mixing a Markov Chain.
		 }   
	\resumeSubHeadingListEnd
	
	
	%-----------EXPERIENCE-----------------
	\section{Professional Experience}
	\resumeSubHeadingListStart
	\resumeSubheading
	{Manchester City Football Club}{Euston, London}
	{Data Analyst -- Fan Relationship Management}{Jan. - July 2018}
	\resumeItemListStart
	
	\resumeItem{New York City FC Project}{}
	{
		
%		This project was a complete mess. I had to work with people from NYCFC who didn't really know where the data was coming from. So I had to talk to uh some third party contractor. What were they called... ? I think they were called \texttt{KORE}.
%		\newline
%		They had all the Ticketmaster data. But we had been given access to one of their remote databases. For some reason, we were using \texttt{Informatica} to query this database, export to some flat file on an FTP and then execute a process (stored procedure) to import this file (spreadsheet) into some staging table in SQL where it was NOT transformed into relational tables.
%		\newline
%		Everything was just a god damn spreadsheet. I've never seen a more disturbing use of SQL in my life.
%		\newline
%		While there was nothing I could do about the structure of the database, I was able to replace the entire \texttt{Informatica} process with a single stored procedure that simply queryied the remote database with an \texttt{OPENQUERY} and a \texttt{MERGE}. Automating this would have been a matter of using \textit{SQL Sever Information Services} but for some reason it would just not work with SSMS 2008. By they time the database had been upgraded to a better version, I was long gone.
		
		I took ownership of this project to integrate \textit{NYCFC's} transactional and demographic data with \textit{City Football Group's} data warehouse. This six-month project involved many phases including: data-discovery, architecture, and data processing. Data came from multiple external sources each with unique schema: (\textit{NYCFC}, \textit{Ticketmaster} \textit{Salesforce}, \textit{Major League Soccer}). 		
	\resumeItemListStart
		
		\resumeItem{Data Pipeline}{}
		{
			I implemented a data pipeline to ingress data from a number of distributed \textit{SQL} databases.
			This process was encapsulated in \textit{stored procedures} which used appropriate DML \& DDL (\texttt{OPENQUERY}, \texttt{MERGE}) for efficient ETL.
			\newline
			This pipeline replaced the slower and more complex \textit{Informatica} solution.
			\newline
		}	
		\resumeItem{Data QA}{}
		{
			For QA I built an aggregated view to compare the distribution of \texttt{NULL} values. To store these analyses, I made use of \textit{Data Cubes} which minimised bandwidth use across \textit{CFG's} distributed databases and improved user experience when explorng these datasets in a \textit{Tableau} front-end as all necessary computations had been performed up-stream.
			\newline
		}
		\resumeItem{Mentoring}{}
		{
			As part of this project, I was dedicated to mentoring a junior colleague remotely in New York. I organised weekly workshops to teach basic DML and more advanced DDL involving process encapsulation and automation. Additional material on my website helped supplement these workshops.
			\newline
		}
	
	\resumeItemListEnd		
	}
	\resumeItem{GDPR Overhaul}{}
	{
		This project involved the refactoring and streamlining of several critical existing production processes to merge GDPR preference data with the analytical database. To accomplish this, SQL DML such as \texttt{merge} was utilized.
		\newline
		I worked closely with the development team to test and produce an automated process to merge these new customer preferences seamlessly into the existing database.
	}
	\resumeItem{Customer Lifetime Value}{}
	{
		Implementation of a discrete-time periodic sales model (blah blah)		
		Churn, SQL data mining. Modeling of LCV via \textit{beta-gaussian, beta-binomial} model.
		Worked closely with data analysts to ensure optimal feature selection.
	}
%	\resumeItem{Tableau Dashboard Automation/Optimization}{}
%	{
%		Implemented \textit{Data Cubes} to pre-aggregate data along all combinations of categorical fields. That is, every possible drill-down and roll-up was computed in advance. As a result, \textit{Tableau} our users would not experience delays due to real-time aggregation.
%	}
%	\resumeItem{Guiding and Mentoring}{}
%	{
%		Instructing junior colleagues on SQL best practices and fundamentals. E.g. understanding DDL \& DML for writing SQL queries and creating database objects; when to return a \href{https://adrian.ng/SQL/joins/cartesian/}{\textit{product join}} vs \href{https://adrian.ng/SQL/joins/semi/}{\textit{semi-join}}; making use of \texttt{information\_schema}; utilizing \textit{SQL Agent} to schedule jobs. 
%	}
	\resumeItemListEnd
	\resumeSubheading
	{ITG Creator (Digital Marketing Agency)}{Westminster, London}
	{Senior CRM Campaign Executive -- SQL Development}{Dec. 2013 - Sept. 2016}
	\resumeItemListStart
	\resumeItem{Virgin Media Segmentation Process}{}
	{
%		Built a number of automated segmentation process using SQL stored procedures for team members to use. Recipient data were imported via \texttt{BULK INSERT}, stored in database tables and indexed (clustered). Segmentation data was output and linked to HTML content to be broadcast to recipients.
	Built an end-to-end segmentation process (data ingress + config file = segmentation data). 
	Close coordination with account executives
	Innovative use of \texttt{XML} + dynamic SQL + \texttt{OPENQUERY} resulted in efficient data fetching from remote server.
	\newline
	}
	\resumeItem{Volkswagen New Client Onboarding}{}
	{
		Work assessment group
		Testing
		Mobile and Email CRM
		Efficient approach using recursive query for regex
		\newline
	}
	\resumeItem{TUI Content Redesign}{}
	{
		Three-month project. Close liasing with client.
		Integration of new HTML from client into existing system.
		Ensuring design is reactive.
		Integrating \textit{SQL} table with \textit{HTML} content dynamically using robust \texttt{TCL} scripts to handle multiple design configurations.
		
		Efforts on this project were recognised by client.
		\newline
	}

	
%	\resumeItem{Recursion}{(\href{https://adrian.ng/SQL/cte/Recursion/}{https://adrian.ng/SQL/cte/Recursion/})}
%	{		
%		Used recursive queries (CTEs) to clean data e.g. removing $n$-number of leading zeros from mobile phone numbers in order to prefix with dialling codes; or splitting strings and mapping into relational format.
%	}
%	\resumeItem{Cross-Server Query Optimisation}{(\href{https://adrian.ng/SQL/misc/openquery-xml}{https://adrian.ng/SQL/misc/openquery-xml})}
%	{
%		Improved cross-server query execution speeds by using \texttt{OPENQUERY}, which transmits a string of SQL for execution on the remote (a live database under constant heavy load). Futher, \textit{Dynamic SQL} was utilized to include XML data in the string. Mapping via a \textit{CTE}, this XML could be transformed into a relational object capable of joining to remote objects. As a result, filtering via join occurs remotely and only a small data set is returned via the \texttt{OPENQUERY}.
%	}
	\resumeItem{Soft Skills}{}
	{
		- Attended inter-departmental work assessment groups and advised on work specifications.
		\newline
		- As senior team member, served as point of contact for clients and colleagues looking to resource our team.
		\newline
		- On occasion I held responsibility for resourcing and managing the team's workload using \textit{Jira}.
	}
	\resumeItemListEnd
	
	% seatwave stuff      
	
	\resumeSubheading
	{Seatwave (now Ticketmaster)}{Moorgate, London}
	{Marketing Analyst Intern -- Commercial Team}{May 2013 - Dec. 2013}
	\resumeItemListStart
	
	\resumeItem{Basic SQL}{}
	{
		{
			In this position I gained my first experience writing database queries in \textit{SQL Server Management Studio}. With basic understanding of \textit{DML} and \textit{DDL}, I was able to query the ticketing and customer databases to extract data for warehousing, analysis, and CRM segmentation. 
		}
	}
%	 \resumeItem{Pricing Analysis}{}
%	 {
%	   Looked at sales data to conduct price projections that, due to the volatility of the market, were otherwise unobtainable by the commercial team. This was done by attempting to model this data via exponential decay in \textit{Excel VBA}.
%	 }
%	 \resumeItem{Spatial Analysis}{}
%	 {
%	   Conducted a spatial analysis of customer postcodes, which I used to provide visualisations in \textit{QGIS} of the areas where our customers lived in relation to the location of upcoming events. These visualisations enabled the commercial team to gain a better picture of where they needed to best target their marketing campaigns.
%	 }
	\resumeItemListEnd
	% 
	% seatwave end      
	\resumeSubHeadingListEnd
	
	
	% %-----------COURSEWORK-----------------
	% \section{Postgraduate Coursework: Implementation of Machine Learning Algorithms}
	% \resumeSubHeadingListStart  
	% \item{
	%   \textbf{R Language}{: \\ \quad k-Nearest Neighbours \quad\quad\quad LDA/QDA\quad\quad\quad Kernel Methods \\ \quad Neural Network regression \quad Decision Trees \quad Hierarchical Clustering}
	% } 
	% \item{
	%   \textbf{MATLAB}{: \\ \quad Hidden Markov Models \quad\quad Aggregating Algorithm}
	% } 
	% \resumeSubHeadingListEnd
	
	%--------PROGRAMMING SKILLS------------
	\section{Technologies}
	\resumeSubHeadingListStart
	\item{
		\textbf{Languages:}{\hfill Java 8, T-SQL}	
		\vspace{-5pt}	
	}
	\item{
		\textbf{Software:}{\hfill IntelliJ IDEA, SQL Server Management Studio, Git, Jira, Maven}
		\vspace{-5pt}		
	}
	
%	\item{
%		\textbf{Look at my code:}{\hfill \href{https://github.com/Adrian-Ng}{https://github.com/Adrian-Ng}}
%		\vspace{-5pt}	
%	}
%
%	\item{
%		\textbf{Discussion of my code:}{\hfill \href{https://adrian.ng}{https://adrian.ng}}
%		\vspace{-5pt}
%	}
	
	\resumeSubHeadingListEnd
	
	
	%-------------------------------------------
\end{document}
%