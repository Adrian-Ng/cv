%-------------------------
% Resume in Latex
% Author : Sourabh Bajaj
% License : MIT
%------------------------

\documentclass[letterpaper,11pt]{article}

\usepackage{latexsym}
\usepackage[empty]{fullpage}
\usepackage{titlesec}
\usepackage{marvosym}
\usepackage[usenames,dvipsnames]{color}
\usepackage{verbatim}
\usepackage[inline]{enumitem}
\usepackage[pdftex]{hyperref}
\usepackage{fancyhdr}
\usepackage[symbol]{footmisc}
\usepackage{multicol}

\pagestyle{fancy}
\fancyhf{} % clear all header and footer fields
\fancyfoot{}
\renewcommand{\headrulewidth}{0pt}
\renewcommand{\footrulewidth}{0pt}

% Adjust margins
\addtolength{\oddsidemargin}{-0.375in}
\addtolength{\evensidemargin}{-0.375in}
\addtolength{\textwidth}{1in}
\addtolength{\topmargin}{-.5in}
\addtolength{\textheight}{1.0in}
% https://tex.stackexchange.com/questions/53962/why-are-urls-typeset-with-monospace-fonts-by-default
\urlstyle{same}

%%\raggedbottom
%\raggedright
\setlength{\tabcolsep}{0in}
\linespread{0.97}


% Sections formatting
\titleformat{\section}{
	\vspace{-4pt}\scshape\raggedright\large
}{}{0em}{}[\color{black}\titlerule \vspace{-5pt}]

%-------------------------
% Custom commands
\newcommand{\resumeItem}[3]{
	\item\small{
		\textbf{#1}\hfill\tiny{#2\\}\small{ #3 \vspace{-2pt}}
	}
}

\newcommand{\resumeSubheading}[4]{
	\vspace{-1pt}\item
	\begin{tabular*}{0.97\textwidth}{l@{\extracolsep{\fill}}r}
		\textbf{#1} & #2 \\
		\textit{\small#3} & \textit{\small #4} \\
	\end{tabular*}\vspace{-5pt}
}

\newcommand{\resumeSubItem}[3]{\resumeItem{#1}{#2}{#3}\vspace{-2pt}}

\renewcommand{\labelitemii}{$\circ$}

\newcommand{\resumeSubHeadingListStart}{\begin{itemize}[leftmargin=*]}
	\newcommand{\resumeSubHeadingListEnd}{\end{itemize}}
\newcommand{\resumeItemListStart}{\begin{itemize}}
	\newcommand{\resumeItemListEnd}{\end{itemize}\vspace{-5pt}}
	\renewcommand{\thefootnote}{\fnsymbol{footnote}}
%-------------------------------------------
%%%%%%  CV STARTS HERE  %%%%%%%%%%%%%%%%%%%%%%%%%%%%
\begin{document}
%----------HEADING-----------------
\noindent
\Large \textbf{Adrian Ng, MSc.}
\newline
\small
Seeking Junior-Level Data Engineering Opportunities
\hfill
\begin{description*}
	\item [Email:] \href{mailto:contact@adrian.ng}{contact@adrian.ng}
	\item [Website:] \href{https://adrian.ng}{adrian.ng}
\end{description*}
%----------PROFILE-----------------
\section{Profile}
\begin{paragraph}
	I am a Computer Science graduate passionate about programming and a career in Data Engineering. I seek opportunities that meet my growing experience in \textit{Java} -- a language I have used in numerous academic projects ranging from the implementation of financial models to large-scale data processing with \textit{Apache Hadoop}. In addition, \texttt{R} and \texttt{MATLAB} were used to implement various machine learning algorithms.
	\par
	Prior to postgraduate study, my expertise was in \textit{SQL development} focusing on the implementation of segmentation processes for a number of clients including: \textit{Virgin Media}, \textit{TUI}, \textit{UPC}, \textit{MSD}, \textit{Volkswagen}, and \textit{KwikFit}.
	\par
	After graduation, my accomplishments as a Data Analyst at \textit{Manchester City FC} fell more towards Data Engineering, which leads me now to pursue a career in this field.
\end{paragraph}
%-----------EDUCATION-----------------
\section{Education}
\resumeSubHeadingListStart
\resumeSubheading
{Master of Science in Data Science and Analytics}{with Distinction}
{Department of Computer Science, Royal Holloway}{Sept. 2016 -- Dec. 2017}

% \begin{itemize*}
% 	\item[] \tiny \textbf{Java Modules:}
% 	\item \tiny Programming for Data Analysis
% 	\item \tiny Large Scale Data Storage and Processing
% 	\item \tiny Methods of Computational Finance
% 	\item \tiny Dissertation
% \end{itemize*}
\resumeSubheading
{Bachelor of Engineering in Mechanical Engineering}{Upper Second Class with Honours}
{School of Engineering, King's College London}{Sept. 2007 -- July 2010}
\resumeSubHeadingListEnd

%--------PROGRAMMING SKILLS------------
\section{Technologies}
\textbf{Languages:}
\hfill
\begin{itemize*}
	\item Java 8
	\item SQL
\end{itemize*}
\newline
\textbf{Software:}
\hfill
\begin{itemize*}
	\item IntelliJ IDEA
	\item SQL Server Management Studio
	\item Git
	\item Jira
	\item Maven
\end{itemize*}

%-----------PROJECTS-----------------
\section{Java Projects}
%\subsection{VaR}
\begin{itemize}
	\item {
	      \textbf{Implementation of Value at Risk (VaR) measure}
	      \hfill
	      \tiny
	      (\href{https://adrian.ng/java/var/}{https://adrian.ng/java/var/})
	      \hfill
	      (\href{https://github.com/Adrian-Ng/VaR}{https://github.com/Adrian-Ng/VaR})
	      \newline
	      \small
	      I implemented number of approaches to estimating \textit{VaR}, a measure of risk against a (hypothetical) investment portfolio (stocks, options, deltas). Various \textit{VaR} measures were implemented. A Normal Distribution $N(0, \sigma^2)$ was assumed for daily price changes. Therefore, a number of moving average processes were implemented for estimating variance $\sigma^2$.
	      In addition, an implementation of the \textit{Levenberg-Marquardt} algorithm was used for optimisation of \textit{GARCH(1,1)} parameters.
	      %https://tex.stackexchange.com/questions/233780/two-itemized-lists-side-by-side
	      \begin{multicols}{2}
		      \textbf{VaR Measures}
		      \begin{itemize}
			      \item Model Building
			      \item Historical Simulation
			      \item Monte Carlo Simulation.
		      \end{itemize}
		      \columnbreak
		      \textbf{Moving Averages}
		      \begin{itemize}
			      \item \textit{Equal Weighted}
			      \item \textit{Exponentially Weighted Moving Average (EWMA)}
			      \item \textit{GARCH(1,1)}
		      \end{itemize}
	      \end{multicols}
	      To accommodate these numerous approaches, I made use of object-oriented techniques and patterns.
	      In addition, I used Java's concurrency APIs to parallelize the 100,000+ random walks generated by \textit{Monte Carlo} when simulating stock prices.
	      Real world financial data was obtained via \textit{Google Finance} and \textit{Yahoo Finance} APIs.
	      }
	\item
	      {
	      \textbf{Option Pricing}
	      \hfill
	      \tiny
	      (\href{https://adrian.ng/java/options/}{https://adrian.ng/java/options/}) \quad (\href{https://github.com/Adrian-Ng/OptionPricer}{https://github.com/Adrian-Ng/OptionPricer})
	      \newline
	      \small
	      As part of the postgraduate module, \textit{Methods of Computational Fianance}, I implemented three approaches to estimating option prices:
	      \begin{itemize*}
		      \item Monte Carlo Simulation
		      \item Black Scholes
		      \item Binomial Trees.
	      \end{itemize*}
	      And where applicable, I computed the \textit{payoff} for American, Asian, and European options.
	      The Black Scholes approach assumes a Normal Distribution, so Apache Commons Math API was used.
	      }
	\item
	      {
	      \textbf{Data Mining with Hadoop MapReduce}
	      \hfill
	      \tiny
	      (\href{https://github.com/Adrian-Ng/HadoopEnron}{https://github.com/Adrian-Ng/HadoopEnron})
	      \small
	      \newline
	      As part of the postgraduate module \textit{Large Scale Data Storage and Processing}, I wrote a number of \textit{MapReduce} applications for \textit{Hadoop}. These included mapping the communications network from the \textit{Enron Corpus}, a large dataset of emails, via \textit{Regex} and aggregation of Twitter data.
	      \newline
	      Applications were exported and executed on \textit{Hadoop} clusters (both single node and distributed). Input/Output datasets were stored in HDFS and accessed via \texttt{hadoop fs} commands.
	      \newline
	      A subsequent exercise was undertaken to minimise the verbosity of these \textit{Hadoop MapReduce} applications by translating them to \textit{Scala} for use in a \textit{Spark REPL}.
	      }
	\item {
	      \textbf{Java 8 Streams with financial data}
	      \hfill
	      \tiny
	      (\href{https://adrian.ng/java/yahoofinance/\#stream}{https://adrian.ng/java/yahoofinance/\#stream})
	      \small
	      \newline
	      A small exercise involving the use of \textit{Java 8 Streams}. Processing real-world financial data to return \textit{mean} and \textit{variance} of some market asset.
	      }
	      %		 \resumeSubItem{Google PageRank}{}
	      %		 {
	      %		   This is the implementation of Google's \textit{PageRank} algorithm. I simulate the behaviour of someone browsing a series of webpages by computing a transition matrix from an input graph and mixing a Markov Chain.
	      %		 }   
\end{itemize}

% %-----------COURSEWORK-----------------
% \section{Machine Learning Algorithms Implemented}
% \textbf{R}
% \hfill
% \begin{itemize*}
% 	\item k-Nearest Neighbours
% 	\item LDA
% 	\item Neural Networks
% 	\item Decision Trees
% 	\item Hierarchical Clustering
% \end{itemize*}
% \newline
% \textbf{MATLAB}
% \hfill
% \begin{itemize*}
% 	\item Hidden Markov Models
% 	\item Aggregating Algorithm
% \end{itemize*}

\newpage
%-----------EXPERIENCE-----------------
\section{Manchester City Football Club}
\textit{Data Analyst}
\hfill
\textit{Fan Relationship Management}
\hfill
\textit{Jan. - July 2018}
\begin{itemize}
	\item
	      {
	      \textbf{New York City FC Project:}
	      I took ownership of this project to integrate \textit{NYCFC's} transactional and demographic data with \textit{City Football Group's} data-warehouse. This six-month project involved many phases including: discovery, engineering, and analysis. Data came from multiple external sources each with differing schema: \textit{NYCFC}, \textit{Ticketmaster} \textit{Salesforce}, \textit{Major League Soccer}.
	      \begin{itemize}
		      \item
		            {
		            \textbf{Data Pipeline:}
		            I implemented a data pipeline to ingress data from a number of remote \textit{SQL} databases.
		            This process was encapsulated in \textit{stored procedures} which used appropriate DML \& DDL (\texttt{OPENQUERY}, \texttt{MERGE}) for efficient ETL.
		            This pipeline replaced the slower front-end  \textit{Informatica} solution.
		            }
		      \item
		            {
		            \textbf{Data Cubes:}
		            I used an aggregated dataset to compare the distribution of \texttt{NULL} values. These analyses were transformed to \textit{Data Cubes} to pre-compute every possible roll-up/drill-down. As such, bandwidth was minimised across our distributed servers and need for real-time computation in \textit{Tableau} front-end was eliminated, resulting in improved user-experience.
		            }
		      \item
		            {
		            \textbf{Mentoring:}
		            As part of this project, I was dedicated to mentoring a junior colleague remotely in New York. I organised weekly workshops to teach basic DML and more advanced DDL with a goal toward self-sufficiency in writing database queries and working with stored procedures. Additional material on my website helped supplement these workshops.
		            }
	      \end{itemize}
	      }
	\item
	      {
	      \textbf{GDPR Stream Integration:}
	      I worked on the integration of a GDPR preference stream into our data stores (\textit{SQL}, \textit{Salesforce}). I implemented a new pipeline and refactored numerous processes downstream .
	      I worked with the development team to provided specification and UAT testing. I built an efficient, automated \texttt{MERGE} process using primary key constraints, clustered indexes, triggers.
	      }
	\item
	      {
	      \textbf{Customer Churn Model:}
	      I contributed datasets and collaborated on feature/model selection. In particular, looking at \textit{logistic regression} and \textit{Beta-Geometric/Beta-Bernoulli} models in R Studio.
	      }
\end{itemize}
\section{ITG Creator}
\textit{Senior CRM Campaign Executive}
\hfill
\textit{SQL Development}
\hfill
\textit{Dec. 2013 - Sept. 2016\\}

\noindent
The majority of my work in this role involved working with SQL processes which were used to transform customer data into CRM segmentations. Having been promoted to the senior position, I developed a number of these processes. On occasion, I held responsibility for resourcing and managing the team's workload using \textit{Jira}.

\begin{itemize}
	\item
	      {
	      \textbf{Virgin Media Segmentation}
	      \hfill
	      \tiny
	      (\href{https://adrian.ng/SQL/cte/Recursion/}{https://adrian.ng/SQL/cte/Recursion/}
	      \quad
	      (\href{https://adrian.ng/SQL/misc/openquery-xml}{https://adrian.ng/SQL/misc/openquery-xml})
	      \newline
	      \small
	      I built an end-to-end segmentation process in \textit{SQL}. This included building a fast, flexible, and bespoke import tool around \texttt{BULK INSERT}. Remote server queries (\texttt{OPENQUERY}) made use of \texttt{XML} to effectively \texttt{INNER JOIN} local and remote tables resulting in speed and minimial resource use on a busy live server. Recursive queries were used to implement a solution (similar to \texttt{flatMap} in \textit{Java 8}) for efficient \textit{regex}.
	      }
	\item
	      {
	      \textbf{Volkswagen Onboarding:}
	      I worked with \texttt{.NET} developers and project managers to bring Volkswagen on-board as a new client. This required implementing a new segmentation process for broadcasting email \textit{and} SMS. In addition, I provided specification to developers for their data warehousing/archiving ingress schema.
	      }
	\item
	      {
	      \textbf{TUI Redesign:}
	      I collaborated closely with the TUI client during a three-month project to redesign the existing \textit{Thomson} and \textit{First Choice} mailings.	\texttt{TCL} scripts were developed to merge dynamic content into the \texttt{HTML} body. My efforts on this project were awarded by the client.
	      }
\end{itemize}
\section{Seatwave (now Ticketmaster)}
\textit{Marketing Analyst Intern}
\hfill
\textit{Commercial Team}
\hfill
\textit{May 2013 - Dec. 2013\\}

\noindent
Using \textit{SQL Server Management Studio} for the first time, I wrote \textit{DML} capable of querying the transactional/customer databases to return data for warehousing, reporting, and segmentation. I also worked on pricing and spatial analyses, using \textit{QGIS} as a visualisation tool.
\end{document}