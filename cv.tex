\documentclass[letterpaper,11pt]{article}
\usepackage{latexsym}
\usepackage[empty]{fullpage}
\usepackage{titlesec}
\usepackage{marvosym}
\usepackage[usenames,dvipsnames]{color}
\usepackage{verbatim}
\usepackage[inline]{enumitem}
% change pdftex to tex4ht when outputting odt file!
\usepackage[pdftex]{hyperref}
\usepackage{fancyhdr}
\usepackage[symbol]{footmisc}
\usepackage{multicol}
\usepackage[mark]{gitinfo2}

\pagestyle{fancy}
\fancyhf{} % clear all header and footer fields
\fancyfoot{}
\renewcommand{\headrulewidth}{0pt}
\renewcommand{\footrulewidth}{0pt}

% Adjust margins
\addtolength{\oddsidemargin}{-0.375in}
\addtolength{\evensidemargin}{-0.375in}
\addtolength{\textwidth}{1in}
\addtolength{\topmargin}{-.5in}
\addtolength{\textheight}{1.0in}
% https://tex.stackexchange.com/questions/53962/why-are-urls-typeset-with-monospace-fonts-by-default
\urlstyle{same}

%%\raggedbottom
%\raggedright
\setlength{\tabcolsep}{0in}
\linespread{0.97}


% Sections formatting
\titleformat{\section}{
	\vspace{-4pt}\scshape\raggedright\large
}{}{0em}{}[\color{black}\titlerule \vspace{-5pt}]



% gitinfo2

\renewcommand{\gitMark}{
Branch: \gitBranch\,@\,\gitAbbrevHash{}
\textbullet{}
Release:\gitReln{}(\gitAuthorDate)
\textbullet{}
Latest: \href{https://adrian.ng/cv}{adrian.ng/cv}
}

%-------------------------

%-------------------------------------------
%%%%%%  CV STARTS HERE  %%%%%%%%%%%%%%%%%%%%%%%%%%%%
\begin{document}
%----------HEADING-----------------
\noindent
\Large
\textbf{Adrian Ng, MSc.}
\newline
\small
Seeking Junior-Level Data Engineering Opportunities
\newline
\begin{description*}
	\item[Email:] \href{mailto:contact@adrian.ng}{contact@adrian.ng}
	\item[Website:] \href{https://adrian.ng}{adrian.ng}
	\item[Location:] London
\end{description*}
\hfill
\vspace{0.25cm}


\noindent
\begin{minipage}[t]{0.5\linewidth}
	\section{Profile}
	\quad
	I am a Computer Science graduate passionate about programming and a career in Data Engineering.
	\par\quad
	I seek opportunities that meet my growing experience in \textit{Java} -- a language I have used in numerous academic projects ranging from the implementation of financial models to large-scale data processing with \textit{Apache Hadoop} and more.
	%In addition, \texttt{R} and \texttt{MATLAB} were used to implement various machine learning algorithms.
	\par\quad
	My professional expertise in \textit{SQL development} focuses on the implementation of segmentation processes for a number of clients including:
	\begin{itemize*}
		\item Virgin Media
		\item TUI
		\item UPC
		\item MSD
		\item Volkswagen
		\item KwikFit
	\end{itemize*}
	\par\quad
	And with my technical and data-oriented accomplishments at \textit{Manchester City FC}, I now pursue a career in the field of Data Engineering.
\end{minipage}
\hspace{0.2cm}
\begin{minipage}[t]{0.3\linewidth}
	\section{Education}
	\textbf{Royal Holloway} \hfill 2017
	\newline
	\textit{Data Science \& Analytics}
	\begin{itemize}[leftmargin=*, itemsep=0.1em]
		\item	Master of Science
		\item	Pass with Distinction
		\item 	Department of Computer Science
	\end{itemize}
	\vspace{0.15cm}
	\textbf{King's Collge London} \hfill 2010
	\newline
	\textit{Mechanical Engineering}
	\begin{itemize}[leftmargin=*, itemsep=0.1em]
		\item Bachelor of Engineering
		\item Upper Second Class with Honours
		\item School of Engineering
	\end{itemize}
\end{minipage}
\hspace{0.2cm}
\begin{minipage}[t]{0.15\linewidth}
	\section{Languages}
	\begin{itemize}[leftmargin=*]
		\item Java 8
		\item T-SQL
	\end{itemize}
	\section{Software}
	\begin{itemize}[leftmargin=*]
		\item IntelliJ IDEA
		\item MS SQL Server
		\item VS Code
		\item Git
		\item Jira
		\item Maven
	\end{itemize}
\end{minipage}

%-----------PROJECTS-----------------
\section{Java Projects}
%DISSERTATION
\begin{description}[style=multiline,leftmargin=4cm]

	\item[Value at Risk \textnormal{Dissertation} \textnormal{\tiny
		      \href{https://adrian.ng/java/var/}{adrian.ng/java/var/}}]
	      Estimating \textit{VaR}, a measure of risk, for an investment portfolio containing stocks, options, deltas.

	      \begin{minipage}[b]{0.2\textwidth}
		      \vspace{0.25cm}
		      %   \begin{description}[style=multiline,leftmargin=4cm]
		      \textbf{VaR Measures}
		      \begin{itemize}[leftmargin=*]
			      \tiny
			      \item Model Building
			      \item Historical Simulation
			      \item Monte Carlo Simulation.
		      \end{itemize}
		      \textbf{Volatility Estimates}
		      \begin{itemize}[leftmargin=*]
			      \tiny
			      \item \textit{Equal Weighted}
			      \item \textit{Equal Weighted Moving Average (EWMA)}
			      \item \textit{GARCH(1,1)}
		      \end{itemize}
		      %   \end{description}
		      %   \vspace{0.25cm}
	      \end{minipage}
	      \hspace{0.25cm}
	      \begin{minipage}[b]{0.55\textwidth}

		      \begin{itemize}[leftmargin=*]
			      \item implemented \textit{Levenberg-Marquardt} algorithm for optimisation of \textit{GARCH(1,1)} parameters
			      \item made use of object-oriented techniques and patterns
			      \item gained efficiencies using Java's \texttt{concurrency} APIs to parallelize the 100,000+ random walks generated by \textit{Monte Carlo} when simulating stock price movements
			      \item utilised \textit{Google Finance}/\textit{Yahoo Finance} APIs to source time-series financial data
		      \end{itemize}
	      \end{minipage}

	      \dotfill
	      %BIG DATA
	\item[Data Mining \textnormal{Large-Scale Data Storage \& Processing} \textnormal{\tiny\href{https://github.com/Adrian-Ng/HadoopEnron}{adrian.ng/java/enron}} \textnormal{\tiny
		      \href{https://adrian.ng/scala/enron1}{adrian.ng/scala/enron1}}]
	      %   Wrote \textit{MapReduce} applications for large scale data mining and processing.
	      \begin{description}[style=multiline,leftmargin=2.5cm]
		      \item[MapReduce]
		            Wrote \textit{MapReduce} applications involving:
		            \begin{itemize}
			            \item aggregation of \textit{Twitter} data
			            \item scraping a large collection of emails in the \textit{Enron Corpus}
			            \item extraction of communications graph consisting of nodes/edges
		            \end{itemize}
		      \item[Hadoop]
		            \begin{itemize}
			            \item Applications ran on both single-node (self-hosted)/ distributed-node clusters
			            \item Interfaced with \textit{HDFS} via terminal command-line
		            \end{itemize}
		      \item[Spark] Utilised an \textit{Apache Spark REPL} to achieve:
		            \begin{itemize}
			            \item translation of \textit{MapReduce} applications to \texttt{Scala}
			            \item reduced code verbosity
			            \item ETL via \textit{HDFS} using \texttt{sparkcontext} API
		            \end{itemize}
	      \end{description}
	      \dotfill
	      %   \vspace{0.25cm}
	\item[Option Pricing \textnormal{Methods of Computational Finance} \textnormal{\tiny\href{https://adrian.ng/java/options/}{adrian.ng/java/options/}}]
	      Implemented numerous approaches to pricing options and calculating payoff:
	      \begin{description}[style=multiline,leftmargin=2.85cm]
		      \item[Options]
		            \begin{itemize*}
			            \item Monte Carlo Simulation
			            \item Black Scholes
			            \item Binomial Trees
		            \end{itemize*}
		      \item[Payoff]
		            \begin{itemize*}
			            \item American
			            \item Asian
			            \item European
		            \end{itemize*}
	      \end{description}
	      These approaches made probabilistic assumptions, so \textit{Apache Commons Math} API was used.

	      \dotfill

	\item[Summarizing financial data \newline \textnormal{\tiny
		      \href{https://adrian.ng/java/yahoofinance/}{adrian.ng/java/yahoofinance/}}]
	      An exercise in using Java 8's \texttt{Stream} API.
	      I was able to implement approaches to computing mean and variance estimates from an immutable collection of time-series financial data.

	      %   \begin{itemize}
	      %       \item implmented approaches to computing statistics (mean, variance) on time-series financial data
	      %   \end{itemize}
	      %       \vspace{0.5cm}
	      % \item[Google PageRank]
	      %       Implementation of Google's \textit{PageRank} algorithm. I simulate the behaviour of someone browsing a series of webpages by computing a transition matrix from an input graph and mixing a Markov Chain.

	      \dotfill

	\item[Webpage Scraping]
	      Wrote a program for scraping data from webpages.
	      \begin{itemize}
		      \item Utilised \texttt{HtmlUnit} and \texttt{Selenium} APIs
		      \item Traversed DOM and parsed child elements via \texttt{xPath}
	      \end{itemize}
\end{description}

\newpage
%-----------EXPERIENCE-----------------
\section{Manchester City Football Club}
\textit{Data Analyst}
\hfill
\textit{Fan Relationship Management}
\hfill
\textit{Jan. - July 2018}

\begin{description}[style=multiline,leftmargin=3cm]
	\item[NYCFC Data Integration \textnormal{Project Owner}]

	      Integrated New York City FC  data into our analytical warehouse.
	      Six-month project involving data discovery, analysis, engineering.
	      Multiple data sources were involved:
	      \begin{itemize*}
		      \item NYCFC
		      \item Ticketmaster
		      \item Salesforce
		      \item Major League Soccer
	      \end{itemize*}
	      \begin{description}[style=multiline,leftmargin=2.5cm]
		      \item[Data Pipeline]
		            {
		            \begin{itemize}
			            \item built pipeline ingesting data from multiple databases, replacing \textit{Informatica} solution
			            \item achieved speed improvements using efficient DML \& DDL (\texttt{OPENQUERY}, \texttt{MERGE})
		            \end{itemize}
		            }

		      \item[Data Cubes]
		            Storing analytical datasets in \textit{Data Cubes} achieved
		            \begin{itemize}
			            \item up-stream computation of all drill-down/roll-up levels and \texttt{GROUP BY} permutations
			            \item reduction in size of dataset, minimising bandwidth across distributed servers
			            \item improved user-experience in \textit{Tableau} front-end
		            \end{itemize}

		      \item[Mentoring]
		            Dedicating time to mentoring junior colleagues remotely in Manchester/New York
		            \begin{itemize}
			            \item organised weekly workshops teaching basic DML and advanced DDL
			            \item developed additional material on my website to supplement these workshops
			            \item aimed towards self-sufficiency in writing database queries/stored procedures
		            \end{itemize}

	      \end{description}
	\item[GDPR Pipeline \textnormal{Technical Lead}]
	      \begin{itemize}
		      \item integrated new GDPR schema into existing datastores (\textit{SQL}, \textit{Salesforce})
		      \item provide schema specification to \texttt{SQL} developers, advocating for indexable data types
		      \item built efficient \texttt{MERGE} process featuring relational database design
		      \item implemented a process to wipe personalised data belonging to any non-consenting individual stored in our data-warehouse
	      \end{itemize}
	\item[Customer Churn Model]
	      Modelling MCFC/NYCFC customers' future propensity to churn via \textit{logistic regression}.
	      \begin{itemize}
		      \item contributed to feature selection via:
		            \begin{itemize*}
			            \item data extraction
			            \item imputation
			            \item normalisation
			            \item \texttt{R} modelling
		            \end{itemize*}
		      \item researched alternate models (e.g. \textit{Beta-Geometric/Beta-Bernoulli}), academic papers, \texttt{R} APIs
	      \end{itemize}
\end{description}
\section{Creator (now Inspired Thinking Group)}
\textit{Senior CRM Campaign Executive}
\hfill
\textit{SQL Development}
\hfill
\textit{Dec. 2013 - Sept. 2016}
\begin{paragraph}
	I developed a number of \texttt{SQL} processes to transform customer data into CRM segments. On occasion, I took responsibility for resourcing and managing the team's workload in \textit{Jira}.
\end{paragraph}
\begin{description}[style=multiline,leftmargin=3cm]
	\item[Virgin Media Segmentation \tiny\textnormal{\href{https://adrian.ng/SQL/recursion}{adrian.ng/SQL/recursion}} \textnormal{\href{https://adrian.ng/openquery-xml}{adrian.ng/openquery-xml}}]
	      Built a flexible segmentation process able to accommodate the numerous VM mailings and myriad ad-hoc configurations.
	      \begin{itemize}
		      \item wrote a flexible import process to efficiently ingest millions of tuples distributed across multiple flat-files, gaining time-savings over the built-in import wizard
		      \item achieved efficient joining of local and remote tables via use of \texttt{OPENQUERY}, \texttt{XML}, dynamic \texttt{SQL}
		      \item implemented efficient regex parsing via recursion, producing a one-to-many tuple mapping
	      \end{itemize}
	\item[Volkswagen Onboarding]
	      Worked with \texttt{.NET} developers and project managers to on-board a new client.
	      \begin{itemize}
		      \item built and tested a new process for segmenting email \textit{and} SMS from scratch
		            %TODO
		            %something about direct mail and business logic
		      \item provided schema specification to developers for data warehousing
	      \end{itemize}
	\item[TUI Redesign]
	      Collaborated with TUI to integrate a new, responsive design for \textit{Thomson} and \textit{First Choice} large deployment broadcasts (5M+ recipients)
	      \begin{itemize}
		      \item wrote \texttt{TCL} scripts for dynamic \texttt{HTML} merges and gained efficiencies by moving expensive operations upstream
		      \item provided testing; gave feedback; managed expectations on technical feasibilites
		      \item gained recognition with client and was awarded at the end of this three-month project
	      \end{itemize}
\end{description}

\section{Seatwave (now Ticketmaster)}
\textit{Marketing Analyst Intern}
\hfill
\textit{Commercial Team}
\hfill
\textit{May 2013 - Dec. 2013\\}


% \begin{itemize}
% 	\item wrote DML to query databases, returning data for
% 	      \begin{itemize*}
% 		      \item warehousing
% 		      \item reporting
% 		      \item segmentation
% 	      \end{itemize*}
% 	\item wrote a \texttt{VBA} macro to fit a line against time-series ticket prices
% \end{itemize}
\noindent
Using \textit{SQL Server Management Studio}, I wrote DML capable of querying the transactional/customer databases to return data for warehousing, reporting, and segmentation. I also worked on pricing and spatial analyses, using \textit{QGIS} as a visualisation tool.

% %-----------COURSEWORK-----------------
% \section{Machine Learning Algorithms Implemented}
% \textbf{R}
% \hfill
% \begin{itemize*}
% 	\item k-Nearest Neighbours
% 	\item LDA
% 	\item Neural Networks
% 	\item Decision Trees
% 	\item Hierarchical Clustering
% \end{itemize*}
% \newline
% \textbf{MATLAB}
% \hfill
% \begin{itemize*}
% 	\item Hidden Markov Models
% 	\item Aggregating Algorithm
% \end{itemize*}
\end{document}