%-------------------------
% Resume in Latex
% Author : Sourabh Bajaj
% License : MIT
%------------------------

\documentclass[letterpaper,11pt]{article}

\usepackage{latexsym}
\usepackage[empty]{fullpage}
\usepackage{titlesec}
\usepackage{marvosym}
\usepackage[usenames,dvipsnames]{color}
\usepackage{verbatim}
\usepackage[inline]{enumitem}
\usepackage[pdftex]{hyperref}
\usepackage{fancyhdr}
\usepackage[symbol]{footmisc}

\pagestyle{fancy}
\fancyhf{} % clear all header and footer fields
\fancyfoot{}
\renewcommand{\headrulewidth}{0pt}
\renewcommand{\footrulewidth}{0pt}

% Adjust margins
\addtolength{\oddsidemargin}{-0.375in}
\addtolength{\evensidemargin}{-0.375in}
\addtolength{\textwidth}{1in}
\addtolength{\topmargin}{-.5in}
\addtolength{\textheight}{1.0in}

\urlstyle{same}

\raggedbottom
\raggedright
\setlength{\tabcolsep}{0in}
\linespread{0.97}


% Sections formatting
\titleformat{\section}{
	\vspace{-4pt}\scshape\raggedright\large
}{}{0em}{}[\color{black}\titlerule \vspace{-5pt}]

%-------------------------
% Custom commands
\newcommand{\resumeItem}[3]{
	\item\small{
		\textbf{#1}\hfill\tiny{#2\\}\small{ #3 \vspace{-2pt}}
	}
}

\newcommand{\resumeSubheading}[4]{
	\vspace{-1pt}\item
	\begin{tabular*}{0.97\textwidth}{l@{\extracolsep{\fill}}r}
		\textbf{#1} & #2 \\
		\textit{\small#3} & \textit{\small #4} \\
	\end{tabular*}\vspace{-5pt}
}

\newcommand{\resumeSubItem}[3]{\resumeItem{#1}{#2}{#3}\vspace{-2pt}}

\renewcommand{\labelitemii}{$\circ$}

\newcommand{\resumeSubHeadingListStart}{\begin{itemize}[leftmargin=*]}
	\newcommand{\resumeSubHeadingListEnd}{\end{itemize}}
\newcommand{\resumeItemListStart}{\begin{itemize}}
	\newcommand{\resumeItemListEnd}{\end{itemize}\vspace{-5pt}}


	\renewcommand{\thefootnote}{\fnsymbol{footnote}}
%-------------------------------------------
%%%%%%  CV STARTS HERE  %%%%%%%%%%%%%%%%%%%%%%%%%%%%


\begin{document}

%----------HEADING-----------------
\begin{tabular*}{\textwidth}{l@{\extracolsep{\fill}}r}
	\textbf{{\Large Adrian Ng MSc.}} & \textbf{Email:} \href{mailto:contact@adrian.ng}{contact@adrian.ng} \\
	Seeking Junior-Level Data Engineering Opportunities & \textbf{Website}: \href{https://adrian.ng}{adrian.ng} \\
\end{tabular*}

\section{Profile}
\begin{paragraph}
	I am a Computer Science graduate passionate about Data Engineering and I seek opportunities that meet my growing experience in \textit{Java} -- a language I have used in numerous academic projects ranging from the implementation of  models pulling real-world financial data to large-scale data processing with \textit{Apache Hadoop MapReduce}.
	\par
	Prior to postgraduate study, my expertise was in \textit{SQL development} focusing on the implementation of segmentation processes for a number of clients including: \textit{Virgin Media}, \textit{TUI}, \textit{UPC}, \textit{MSD}, \textit{Volkswagen}, \textit{KwikFit}.
	\par
	After graduation, my most recent accomplishments as a Data Analyst at \textit{Manchester City FC} were in the technical parts (e.g. pipelines, architecture) of the projects I worked on, which leads me now to pursue a career in Data Engineering.
\end{paragraph}
%-----------EDUCATION-----------------
\section{Education}
\resumeSubHeadingListStart
\resumeSubheading
{Master of Science in Data Science and Analytics}{with Distinction}
{Department of Computer Science, Royal Holloway, University of London}{Sept. 2016 -- Dec. 2017}
\resumeSubheading
{Bachelor of Engineering in Mechanical Engineering}{Upper Second Class with Honours}
{School of Engineering, King's College London, University of London}{Sept. 2007 -- July 2010}
\resumeSubHeadingListEnd

%	\section{Postgraduate Modules}
%	\resumeSubHeadingListStart
%		\resumeSubheading
%		{Programming for Data Analysis}{}
%		{Java -- Eclipse etc}{}
%		\resumeSubheading
%		{Large Scale Data Storage and Processing}{}
%		{Java -- Apache Hadoop}{}
%		\resumeSubheading
%		{Methods of Financial Computation}{}
%		{Java, MATLAB}{}
%	\resumeSubHeadingListEnd

% \begin{itemize*}
% 	\item bananas
% 	\item apples
% 	\item oranges and
% 	\item lemons.
% \end{itemize*}

%-----------PROJECTS-----------------
\section{Java Projects}
\resumeSubHeadingListStart
\resumeSubItem
{Implementation of Value at Risk (VaR) measures in Java}
{
	(\href{https://adrian.ng/java/var/}{https://adrian.ng/java/var/})
	\quad
	(\href{https://github.com/Adrian-Ng/VaR}{https://github.com/Adrian-Ng/VaR})
}
{
	Assuming a number of hypothetical investment portfolios, my dissertation project implemented a number of approaches to estimating \textit{VaR}, a measure of risk, and variance/volatility (for model parameterization).
	\begin{itemize}
		\item {
		      \textbf{VaR:}
		      \begin{enumerate*}[label={\alph*)}]
			      \item \textit{Model Building}
			      \item \textit{Historical Simulation}
			      \item \textit{Monte Carlo Simulation}.
		      \end{enumerate*}
		      }
		\item {
		      \textbf{Variance/Volatility:}
		      \begin{enumerate*}[label={\alph*)}]
			      \item \textit{Equal Weighted}
			      \item \textit{Exponentially Weighted Moving Average}
			      \item \textit{GARCH(1,1)}.
		      \end{enumerate*}
		      }
	\end{itemize}
	Because of these numerous approaches, object-oriented techniques and patterns were implemented.
	In addition, I used Java's concurrency APIs to parallelize the 100,000+ random walks generated by \textit{Monte Carlo} for simulating stock prices.
	Real world financial data was obtained via the Google Finance/Yahoo Finance APIs.
}
% \begin{itemize}
% 	\item[]{
% 	      This dissertation project implemented a number of approaches to estimating \textit{VaR}, a measure of risk, and variance/volatility (for model parameterization).
% 	      \begin{itemize}
% 		      \item {
% 		            \textbf{VaR:}
% 		            \begin{enumerate*}[label={\alph*)}]
% 			            \item \textit{Model Building}
% 			            \item \textit{Historical Simulation}
% 			            \item \textit{Monte Carlo Simulation}.
% 		            \end{enumerate*}
% 		            }
% 		      \item {
% 		            \textbf{Variance/Volatility:}
% 		            \begin{enumerate*}[label={\alph*)}]
% 			            \item \textit{Equal Weighted}
% 			            \item \textit{Exponentially Weighted Moving Average}
% 			            \item \textit{GARCH(1,1)}.
% 		            \end{enumerate*}
% 		            }
% 	      \end{itemize}
% 	      }
% 	\item {
% 	      As we had a number of approaches to returning estimates, object oriented techniques and patterns were implemented.
% 	      }
% 	\item {
% 	      The Monte Carlo approach generates the 100,000+ random walks used to simulate stock prices. I used Java's concurrency APIs resulting in a highly efficient solution.
% 	      }
% 	\item {
% 	      To test a hypothetical investment portfolio (stocks, options, deltas), real-world market data was sourced using the \textit{Yahoo Finance API}. A distribution of historical daily price changes was computed and used to estimate model parameters.
% 	      }
% \end{itemize}

\resumeSubItem{Option Pricing}
{
	(\href{https://adrian.ng/java/options/}{https://adrian.ng/java/options/}) \quad (\href{https://github.com/Adrian-Ng/OptionPricer}{https://github.com/Adrian-Ng/OptionPricer})
}
{
	This project implements three approaches to estimating option prices in Java: \textit{Monte Carlo simulation}, \textit{Black-Scholes equations}, and \textit{Binomial Trees}.	Apache Commons Math API was used to deal with some probabilistic assumptions.
}

\resumeSubItem{Data Mining with Hadoop MapReduce}
{
	(\href{https://github.com/Adrian-Ng/HadoopEnron}{https://github.com/Adrian-Ng/HadoopEnron})
}
{
	A number of \textit{MapReduce} applications were written in Java with a variety of purposes including extracting the communications network from the \textit{Enron Corpus}, a large dataset of emails, or aggregation of Twitter data.
	\newline
	Applications were exported and executed on \textit{Hadoop} clusters (both single node and distributed). Input/Output datasets were stored in HDFS and accessed via \texttt{hadoop fs} commands.
	\newline
	A subsequent exercise was undertaken to minimise the verbosity of these \textit{Hadoop MapReduce} applications by translating them to \textit{Scala} for use in a \textit{Spark REPL}.
}

\resumeSubItem{Java 8 Streams with financial data}
{(\href{https://adrian.ng/java/yahoofinance/\#stream}{https://adrian.ng/java/yahoofinance/\#stream})}
{
	A small exercise involving the use of \textit{Java 8 Streams}. Processing real-world financial data to return \textit{mean} and \textit{equal-weighted variance} of some market asset.
}
%		 \resumeSubItem{Google PageRank}{}
%		 {
%		   This is the implementation of Google's \textit{PageRank} algorithm. I simulate the behaviour of someone browsing a series of webpages by computing a transition matrix from an input graph and mixing a Markov Chain.
%		 }   
\resumeSubHeadingListEnd

% %-----------COURSEWORK-----------------
% \section{Postgraduate Coursework: Implementation of Machine Learning Algorithms}
% \resumeSubHeadingListStart  
% \item{
%   \textbf{R Language}{: \\ \quad k-Nearest Neighbours \quad\quad\quad LDA/QDA\quad\quad\quad Kernel Methods \\ \quad Neural Network regression \quad Decision Trees \quad Hierarchical Clustering}
% } 
% \item{
%   \textbf{MATLAB}{: \\ \quad Hidden Markov Models \quad\quad Aggregating Algorithm}
% } 
% \resumeSubHeadingListEnd

\newpage
%-----------EXPERIENCE-----------------
\section{Professional Experience}
\resumeSubHeadingListStart
\resumeSubheading
{Manchester City Football Club}{Euston, London}
{Data Analyst -- Fan Relationship Management}{Jan. - July 2018}
\begin{itemize}
	\item
	      {
	      \textbf{New York City FC Project:}
	      I took ownership of this project to integrate \textit{NYCFC's} transactional and demographic data with \textit{City Football Group's} data-warehouse. This six-month project involved many phases including: discovery, engineering, and analysis. Data came from multiple external sources each with differing schema: \textit{NYCFC}, \textit{Ticketmaster} \textit{Salesforce}, \textit{Major League Soccer}.
	      \begin{itemize}
		      \item
		            {
		            \textbf{Data Pipeline:}
		            I implemented a data pipeline to ingress data from a number of remote \textit{SQL} databases.
		            This process was encapsulated in \textit{stored procedures} which used appropriate DML \& DDL (\texttt{OPENQUERY}, \texttt{MERGE}) for efficient ETL.
		            This pipeline replaced the slower front-end  \textit{Informatica} solution.
		            }
		      \item
		            {
		            \textbf{Data Cubes:}
		            I used an aggregated dataset to compare the distribution of \texttt{NULL} values. These analyses were transformed to \textit{Data Cubes} to pre-compute every possible roll-up/drill-down. As such, bandwidth was minimised across our distributed servers and need for real-time computation in \textit{Tableau} front-end was eliminated, resulting in improved user-experience.
		            }
		      \item
		            {
		            \textbf{Mentoring:}
		            As part of this project, I was dedicated to mentoring a junior colleague remotely in New York. I organised weekly workshops to teach basic DML and more advanced DDL with a goal toward self-sufficiency in writing database queries and working with stored procedures. Additional material on my website helped supplement these workshops.
		            }
	      \end{itemize}
	      }
	\item
	      {
	      \textbf{GDPR Stream Integration:}
	      I worked on the integration of a GDPR preference stream into our data stores (\textit{SQL}, \textit{Salesforce}). I implemented a new pipeline and refactored numerous processes downstream .
	      I worked with the development team to provided specification and UAT testing. I built an efficient, automated \texttt{MERGE} process using primary key constraints, clustered indexes, triggers.
	      }
	\item
	      {
	      \textbf{Customer Churn Model:}
	      I contributed datasets and collaborated on feature/model selection. In particular, looking at \textit{logistic regression} and \textit{Beta-Geometric/Beta-Bernoulli} models in R Studio.
	      }
\end{itemize}

\resumeSubheading
{ITG Creator (Digital Marketing Agency)}{Westminster, London}
{Senior CRM Campaign Executive -- SQL Development}{Dec. 2013 - Sept. 2016}

\begin{itemize}

	\item[]{
	      The majority of my work in this role involved working with SQL processes which were used to transform customer data into CRM segmentations. As senior team member, I developed a number of these processes. On occasion, I held responsibility for resourcing and managing the team's workload using \textit{Jira}.
	      }
	\item
	      {
	      \textbf{Virgin Media Segmentation Process}
	      \hfill
	      \tiny
	      (\href{https://adrian.ng/SQL/cte/Recursion/}{https://adrian.ng/SQL/cte/Recursion/}
	      \quad
	      (\href{https://adrian.ng/SQL/misc/openquery-xml}{https://adrian.ng/SQL/misc/openquery-xml})
	      \newline
	      \small
	      I built an end-to-end segmentation process in \textit{SQL}. This included building a fast, flexible, and bespoke import tool around \texttt{BULK INSERT}. Remote server queries (\texttt{OPENQUERY}) made use of \texttt{XML} to effectively \texttt{INNER JOIN} local and remote tables resulting in speed and minimial resource use on a busy live server. Recursive queries were used to implement a solution (similar to \texttt{flatMap} in \textit{Java 8}) for efficient \textit{regex}.
	      }
	\item
	      {
	      \textbf{Volkswagen Client Onboarding:}
	      \quad
	      I worked with \texttt{.NET} developers and project managers to bring Volkswagen on-board as a new client. This required implementing a new segmentation process for broadcasting email \textit{and} SMS. In addition, I provided specification to developers for their data warehousing/archiving ingress schema.
	      }
	\item
	      {
	      \textbf{TUI Mailing Redesign:}
	      \quad
	      I collaborated closely with the TUI client during a three-month project to redesign the existing \textit{Thomson} and \textit{First Choice} mailings.	\texttt{TCL} scripting was used to merge dynamic content into the \texttt{HTML} body. My efforts on this project were recognised by the client.
	      }
\end{itemize}

% seatwave stuff      

\resumeSubheading
{Seatwave (now Ticketmaster)}{Moorgate, London}
{Marketing Analyst Intern -- Commercial Team}{May 2013 - Dec. 2013}
\begin{itemize}

	\item[]{
	      In this position I gained my first experience writing database queries in \textit{SQL Server Management Studio}. I wrote \textit{DML} and \textit{DDL} capable of querying the transactional/customer databases to return data for warehousing, reporting, and segmentation.
	      }

	      % \resumeItem{Basic SQL}{}
	      % {
	      % 	{
	      % 			In this position I gained my first experience writing database queries in \textit{SQL Server Management Studio}. With basic understanding of \textit{DML} and \textit{DDL}, I was able to query the transactional/customer databases to return data for warehousing, reporting, and segmentation.
	      % 		}
	      % }
	      %	 \resumeItem{Pricing Analysis}{}
	      %	 {
	      %	   Looked at sales data to conduct price projections that, due to the volatility of the market, were otherwise unobtainable by the commercial team. This was done by attempting to model this data via exponential decay in \textit{Excel VBA}.
	      %	 }
	      %	 \resumeItem{Spatial Analysis}{}
	      %	 {
	      %	   Conducted a spatial analysis of customer postcodes, which I used to provide visualisations in \textit{QGIS} of the areas where our customers lived in relation to the location of upcoming events. These visualisations enabled the commercial team to gain a better picture of where they needed to best target their marketing campaigns.
	      %	 }
\end{itemize}
% 
% seatwave end      
\resumeSubHeadingListEnd



%--------PROGRAMMING SKILLS------------
\section{Technologies}
\resumeSubHeadingListStart
\item{
            \textbf{Languages:}{\hfill Java 8, T-SQL}
            \vspace{-5pt}
      }
\item{
            \textbf{Software:}{\hfill IntelliJ IDEA, SQL Server Management Studio, Git, Jira, Maven}
            \vspace{-5pt}
      }

%	\item{
%		\textbf{Look at my code:}{\hfill \href{https://github.com/Adrian-Ng}{https://github.com/Adrian-Ng}}
%		\vspace{-5pt}	
%	}
%
%	\item{
%		\textbf{Discussion of my code:}{\hfill \href{https://adrian.ng}{https://adrian.ng}}
%		\vspace{-5pt}
%	}

\resumeSubHeadingListEnd


%-------------------------------------------
\end{document}
%